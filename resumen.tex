\chapter{Resumen}


	Este trabajo de tesis presenta un análisis teórico de la expansión de fracciones continuas o Continued Fraction Expansion (CFE) para su utilización en la creación de integradores de orden fraccionario. Se realiza el análisis del error que esta aproximación presenta y se genera una metodología de diseño de acuerdo al orden fraccionario mediante la utilización de un algoritmo recursivo.
	
	Utilizando hardware analógico embebido (FPAA) y por medio del dispositivo NI ELVIS II+ se realiza la implementación física de los integradores de orden fraccionario y se realiza la medición de su respuesta en frecuencia por medio de diagramas de Bode. Utilizando los resultados  se realiza un análisis comparativo teórico contra experimental y se desarrollan gráficas de mérito para facilitar el proceso de diseño posterior.
	
	Finalmente se utilizan las metodologías obtenidas para realizar la implementación de un oscilador caótico de orden fraccionario.
	